\chapter{Einleitung}
\label{cha:Einleitung}

\section{Motivation}

Mit dem Release von Docker wurde der Bereich
der \emph{Development Operations}
leichter zugänglich f\"ur Entwickler mit wenig oder gar keiner
Erfahrungen in der System-Administration.

Was aber bei dem Umgang mit Docker auch klar wurde, ist, dass Docker eben nicht
alle Probleme des Betriebs von Software in der Cloud löst und so eben oft
separate
Lösungen gefunden werden müssen. Dies hat oft zu einer Reihe
unterschiedlicher Lösungen geführt, die nicht immer ineinander
gegriffen haben.

Kubernetes stellt eine Lösung für viele dieser Probleme dar und
ist dementsprechend zukunftsweisend f\"ur dieses Feld.

Auch f\"ur einige der privaten Projekte des Autors w\"urde Kubernetes eine
signifikante Verbesserung bedeuten, was der Anstoß f\"ur die Recherche dieses
Themas war.

Mit dieser Arbeit soll herausgefunden werden, wie eine Migration
einer dieser bereits bestehenden
Infrastrukturen zu Kubernetes funktioniert
und inwiefern diese Sinn macht.

\section{Wissenschaftliche Fragestellung}

Die wissenschaftliche Fragestellung, die in dieser Arbeit behandelt werden
soll, ist:
Wie lässt sich nach aktuellem Stand der Technik eine Migration
zu Kubernetes durchführen?
Welche Komponenten sollen dafür kombiniert werden und welche Schwierigkeiten
sind dabei zu überwinden?
Welchen Mehrwert bietet letztlich das migrierte Cluster?

\section{Rahmen dieser Arbeit}

Als Case-Study soll ein existierendes Open Source Projekt auf Kubernetes migriert
werden.

Die Rahmenbedingungen des Betriebes von Kubernetes unter echten
Produktionsbedingungen k\"onnen nicht immer akkurat nachgestellt werden,
da die Serverkosten zu hoch wären, um sie
im Rahmen dieser Arbeit finanzieren zu können.
Sofern möglich, werden aber auch diese Aspekte simuliert.

Neben der Software \emph{Kubernetes} wird es auch um andere Technologien gehen,
da Kubernetes nicht isoliert betrachtet werden kann. Auch das Zusammenspiel
mit dieser Peripherie-Software soll im Fokus dieser Arbeit liegen.
Diese Arbeit steht also nicht stellvertretend für alle Möglichkeiten,
ein Cluster auf Kubernetes aufzusetzen, sondern, um einen speziellen Fall, der
sich haupts\"achich aus den Technologie-Entscheidungen ergibt.

\section{Aufbau}

Es sollen zu Beginn einige Anforderungen festgelegt werden, die im Ziel-System
erfüllt sein sollen. Diese setzen sich zusammen aus Anforderungen, die generell für
Produktions-Systeme gelten, aber auch Aspekte, die man sich von einem Wechsel zum neuen
System erwartet.

Weiter soll die Migration auf das Kubernetes Cluster,
die hier durchgeführt wurde, vorgestellt werden. Hieraus leiten sich auch
Erkenntnisse über Schwierigkeiten und Best-Practices ab, die in diesem Kapitel
behandelt werden.
Der finale Stand der Implementierung des Clusters
wird anschließen anhand der Anforderungen bewertet.
Abschließend wird diese Arbeit zusammengefasst,
sowie potentielle Weiterentwicklungen in einem Ausblick aufgezeigt.

Relevante Code-Beispiele sind in dieser Arbeit aufgef\"uhrt oder finden sich
auf der CD/SD-Karte, welche dieser Arbeit als Anhang beigef\"ugt wurde.
