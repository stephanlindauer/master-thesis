\chapter{Fazit}
\label{cha:Fazit}

Die Auswahl an Technologien f\"ur diese Umsetzung ist uneingeschr\"ankt
zu empfehlen. Jeder Aspekt in der Administration eines Clusters
konnte so abgedeckt werden.

Ob sich eine Migration lohnt, h\"angt gr\"oßtenteils davon ab, wie
groß (im Sinne von Infrastruktur-Anforderung)
das zu migrierende Projekt ist. Bei kleinen Projekten kann
dieses Setup zus\"atzliche Kosten verursachen, anstatt f\"ur Einsparungen
zu sorgen. Auch sollte bedacht werden, dass die Migration und die Einarbeitung
in dieses neue System, Zeit in Anspruch nehmen wird.

Einschr\"ankungen in der
Flexibilit\"at gegen\"uber dem alten Ansatz konnten keine erkannt werden.
Die Umsetzung eigener L\"osungsans\"atze werden durch dieses Setup eher erleichtert.

Durch die Weiterentwicklung von Kubernetes werden sich voraussichtlich
in naher Zukunft auch Dokumentationsfehler und Qualit\"at der Software
und Hardware-Integration weiter verbessern.

Selbst der Autor des \quotes{The HFT Guy}-Blog, als
starker Kritiker von Docker, stellt fest:
\quotes{In the long-term, Kubernetes is the future. It’s a major breakthrough
(or to be accurate, it’s the final brick that is missing for containers to be a
major [r]evolution in infrastructure management).
The question is not whether to adopt Kubernetes, the
question is when to adopt it?} \cite{hftguy}.

Die Antwort auf die Frage \quotes{when to adopt it?} h\"angt nun davon ab,
inwiefern die vielen Vorteile auch die bestehenden Probleme und die
aufw\"andige Migrierung rechtfertigen, oder,
ob es mehr Sinn macht auf reifere Versionen mit besserer Integration
und Dokumentation zu warten.

Mit dieser Arbeit und dem daraus entstandenen Code wird die Adaption
dieser Technologie erleichtert. Der Leser soll damit ermutigt werden, selbst,
auf Grundlage des hier gemachten Erfahrungen und des entstandenen Codes, diese
Technologien zu nutzen und mit den eigenen Erfahrungen und L\"osungen weiter zur
Verbesserung dieses Ansatzes beizutragen.
