\chapter{Zusammenfassung und Ausblick}
\label{cha:Zusammenfassung und Ausblick}

\section{Zusammenfassung}

\subsection{Migrations-Prozess}
Der Migrations-Prozess hat sich als schwerer und langwieriger herausgestellt,
als das anfänglich geplant war.
Die Hauptgründe dafür lagen vor allem
in folgenden Punkten:

\begin{description}
  \item[Qualität der Dokumentation]
  Die Dokumentation, speziell von CoreOS und Kubernetes, ist nicht besonders
  gut.
  Teilweise sind Sachverhalte oder
  Beispiele falsch beschrieben
  und führen den Leser in die Irre \cite{masteroslo}. Im Rahmen dieser Arbeit sind diverse
  Pull Requests auf die Dokumentationen dieser Software auf Github entstanden, um einige
  dieser Missstände zu beheben.
  Die verantwortlichen Teams arbeiten in der Regel relativ
  schnell, sodass ein Fix oft
  innerhalb von 24 Stunden veröffentlicht wurde.
  Größtes Problem in der Pflege der Dokumentationen ist sicher, dass sich
  die jeweilige Software aktuell stätig verändert, aber auch, dass das die
  Peripherie-Software tut. So fanden sich die größten Dokumentationsfehler
  in der Beschreibung, wie mit den Schnittstellen zwischen den
  Software-Komponenten
  umzugehen ist.
  \item[Feedback Loops]
  In dem Technologie-Stack, der für diese Arbeit gewählt wurde, stellt
  Terraform die Infrastruktur in der Cloud auf. Desto komplexer diese
  Infrastruktur wird und desto länger Abhängigkeitsketten werden, desto
  länger dauert die Initialisierung der Infrastruktur.
  Anschließend m\"ussen sich dann die Komponenten, die auf Kubernetes laufen
  sollen,
  bauen oder herunterladen und danach gestartet werden.
  Anfangs dauerte dieser Vorgang
  noch rund 8 Minuten, während das finale Setup oft über 20 Minuten
  benötigt, um
  betriebsbereit zu sein. Ein \emph{Trial and Error}-Ansatz mit
  \code{terraform apply} und \code{terraform destroy} ist also sehr
  zeitintensiv.
\end{description}

Trotzdem hat sich dieser Weg gelohnt, denn so konnten die Einzelteile des
Systems besser verstanden und auf den eigenen
Use Case ausgerichtet werden.

\subsection{Wirtschaftlichkeit}

Finanziell l\"asst sich zusammenfassend sagen, dass eine Migration zu Kubernetes
erst ab einem gewissen Volumen an Traffic wirklich Sinn macht, denn die
etcd-Server und Mindestanforderung an Kubernetes-Server verursachen fixe
Kosten, die bei dem vorherigen
Setup nicht entstanden sind.

Trotz h\"oherer Kosten bei kleinen Projekten, kann sich eine Migration trotzdem
auszahlen. So zum Beispiel, wenn die Kubernetes Features f\"ur dieses Projekt
einen so hohen Mehrwert bringen, dass \"uber die zus\"atzlichen Kosten
hinweggesehen werden kann, oder, wenn damit gerechnet wird, dass in naher
Zukunft der Traffic weiter ansteigt.

F\"ur dieses spezielle Projekt ist die Wirtschaftlichkeit fraglich, denn einen
großen Kostenvorteil hat die Migration nicht geschaffen und mit allen Features,
die in dieser Arbeit implementiert wurden, sind die Kosten sogar deutlich h\"oher.

So kann festgestellt werden: Desto mehr Traffic und desto gr\"oßer die Fluktuation
des Traffics, desto besser k\"onnen
\quotes{Skaleneffekte} \cite{cloudmigra}
genutzt werden und desto mehr lohnt sich eine
Migration auch finanziell.

Die Firma Monzo beispielsweise konnte durch eine
Migration zu Kubernetes die
Infrastruktur-Kosten um 75\% senken \cite{bankingbackend}.

\subsection{Mehrwert dieses Setups}

Die Anforderungen an dieses Cluster konnten größtenteils erf\"ullt werden.
Insgesamt ist eine sehr robuste L\"osung entstanden.
Durch die Verwendung dieser Kombination an Technologien, ergibt sich ein neues Paradigma
und eine Reihe von Features, die vorher so nicht zur Verf\"ugung standen.

Um zu verdeutlichen, wie viel Nutzen diese Features mit sich bringen, sollen
hier nochmal die wichtigsten Punkte aufgelistet werden.

\paragraph{Terraform:}
\begin{itemize}
  \item Infrastructure as Code
  \item Immutable und deklarative Konfiguration
\end{itemize}

\paragraph{AWS:}
\begin{itemize}
  \item API f\"ur Terraform
  \item Auto Scaling Groups
  \item VPC und Security Groups
\end{itemize}

\paragraph{Docker:}
\begin{itemize}
  \item Paketierung der Applikationen
  \item Speicherung und Verteilung der Images
  \item Runtime f\"ur den Betrieb der Container
  \item Security durch Host-Isolation
  \item Portabilit\"at und Konsistenz durch Abstraktion des Host-Systems
\end{itemize}

\paragraph{CoreOS:}
\begin{itemize}
  \item Spezialisierung auf Container-Betrieb
  \item Verteilter konsistener Key/Value-Store durch \emph{etcd}
  \item Network-Overlay durch \emph{flannel}
  \item Update-Strategie durch \emph{locksmith}
  \item Initialisierung durch \emph{Cloud-Config} und \emph{systemd}
\end{itemize}

\paragraph{Kubernetes:}
\begin{itemize}
  \item Auto Scaling von Pods und Nodes
  \item API-Schnittstelle
  \item Service Discovery
  \item Load Balancing
  \item Logging
  \item Monitoring
  \item Konfigurations-Management
  \item Deklarative Deployments
  \item Rolling Updates
  \item Health Check
  \item Auto Recovery
  \item Einbindung peristenter Datentr\"ager
  \item Namespace-Isolierung
\end{itemize}

Anhand dieser Aufz\"ahlungen soll verdeutlicht werden, dass diese Komponenten
den kompletten Lifecycle des Betriebes eines Clusters in der Cloud abdecken.

Kubernetes stellt damit ein Grundger\"ust zur Verf\"ugung, schr\"ankt
durch dieses aber kaum ein und erlaubt weiterhin viel Flexibilit\"at
in der Ausarbeitung von L\"osungen f\"ur den jeweiligen Use Case.

In der Migration zu Kubernetes ist das Aufsetzen des Clusters
selbst immer noch eines der gr\"oßten Schwierigkeiten
und f\"uhrt entsprechend auch zu einer langsameren Adaption dieser Technologie.

Diese Arbeit hat mit der Ausarbeitung der Installation
mit gegebenen Technologie-Stack weiter dazu beigetragen,
dass die Installation eines solchen Clusters,
besser dokumentiert ist und kann nun als Vorlage f\"ur \"ahnliche
Ans\"atze verwendet werden.

Teile des Codes dieser Ausarbeitung wurden auf Github
ver\"offentlicht \cite{steph}
und haben
von der Ver\"offentlichung am 14. Dezember 2016 bis zum Tag der Einreichung
dieser Arbeit am 8. M\"arz 2017 bereits 12 Stars und 3 Watcher gewonnen.
Zudem ist bekannt, dass zwei Berliner Firmen, Stellen dieses Repositories
f\"ur die eigene Implementierung \"ubernommen haben.
Dies zeigt, welches Interesse aktuell bez\"uglich Kubernetes und dieser
Kombination an Technologien besteht und, dass die erarbeitete Herangehensweise
als Blaupause f\"ur andere Projekte dienen kann.

\section{Ausblick}

Auch, wenn das Ergebnis dieser Arbeit sehr zufriedenstellend ist, gibt es
weitere Punkte, die in Angriff genommen werden sollten, sobald das Cluster, wie
hier beschrieben, aufgesetzt wurde.

\subsection{Fine-tuning der automatischen Skalierung}

Die automatische Skalierung der Pods und der Nodes hat in den Tests gut funktioniert.
Auf lange Sicht werden sich, basierend auf Erfahrungen aus dem Live-Betrieb,
sicherlich noch bessere Einstellungen finden lassen.

\subsection{Konfiguration von Alerting, Logging und Monitoring}

Um die Alerting-, Logging- und Monitoring-Tools in ihrem vollen
Funktionsumfang nutzen zu k\"onnen, sollten diese noch weiter konfiguriert
und an den Use Case angepasst werden.

\subsection{PostgreSQL-Cluster}

Um die Leistungsf\"ahigkeit und Ausfallsicherheit der Datenbank zu
erh\"ohen, k\"onnen weitere Pods integriert werden, die dann als
Read-Slaves oder Hot-Standby fungieren. F\"ur den Use Case dieser
Arbeit war ein solches Setup jedoch (noch) nicht erforderlich.

\subsection{Regions und Availability Zones}

Eine Achillesferse der Ausfallsicherheit dieser Umsetzung ist die Tatsache,
dass es in nur einer \emph{Availability Zone} in der Frankfurt \emph{Region} deployed
wird. Sollte diese \emph{Availability Zone} oder die ganze \emph{Region}
ausfallen, w\"urde damit auch
das Cluster ausfallen.
Je nach Anspruch an die Ausfallsicherheit eines solchen Clusters sollte in Betracht
gezogen werden, das Cluster \"uber mehrere \emph{Availability Zones}
und eventuell sogar \emph{Regions} (in Extremf\"allen auch IaaS-Provider) auszubreiten
\cite{arch4scale}.
Aufgrund der zeitlichen und finanziellen Rahmenbedingungen
dieser Arbeit, ist dies jedoch nicht umgesetzt worden.
