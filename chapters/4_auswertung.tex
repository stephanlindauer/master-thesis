\chapter{Auswertung}
\label{cha:Auswertung}


\section{Kosten}

Der Vergleich der Kosten stellt sich als relativ komplex dar.
Das Cluster bringt eine Reihe von
Vorteilen mit sich, die monetär nicht zu beziffern sind.
Was sich auch durch die Migration grundlegend verändert hat, ist die Struktur der
Kosten.
Durch die automatische
Skalierung konnten einige der fixen Kosten in variable
Kosten umgewandelt werden.
Diese fallen nur dann an, wenn auch wirklich die entsprechende
Nachfrage besteht.

Im Vergleich der Kosten geht es also im Wesentlichen darum, den alten Ansatz mit
größtenteils fixen Kosten mit dem neuen Ansatz mit fixen \textbf{und} variablen Kosten
zu vergleichen.

Der alte Ansatz hat Kosten von \$ 80.- im Monat verursacht, die nur knapp noch die
anfallende Last verarbeiten konnten. Wäre hier die Last auch nur geringfügig
gestiegen, wäre eine leistungsfähigere Instanz nötig geworden,
welche bei Digital Ocean dann doppelt so viel Leistung bringt,
aber auch doppelt so viel, also \$ 160.-, kostet.
Hieran ist auch zu erkennen, dass eines der Hauptvorteile des neuen Systems ist, dass
es fast stufenlos skalieren kann, während bei der alten Herangehensweise
manuelle und relativ
große Leistungs- und Kostensprünge gemacht werden müssen, um weiterhin
die monatlichen Lastobergrenzen abdecken zu können.

In folgender Tabelle werden die Kosten für das Cluster, so wie es im
Rahmen dieser Arbeit umgesetzt wurde, berechnet.

\begin{flushleft}
\centering
\begin{tabular}{ | l | l | l | l | }
  \hline
  Zweck          & Produkt & Anzahl & Monatliche Kosten \\ \hline \hline
  Master         & EC2 t2.medium  & 1 & \$ 39.53 \\ \hline
  Worker         & EC2 t2.small   & 2 (Minimum) & \$ 39.54 \\ \hline
  Ingress Worker & EC2 t2.medium  & 1 & \$ 39.53 \\ \hline
  Admin Worker   & EC2 t2.medium  & 1 & \$ 39.53 \\ \hline
  Testing Worker & EC2 t2.small   & 1 & \$ 19.77 \\ \hline
  etcd           & EC2 t2.nano    & 3 & \$ 14.94 \\ \hline
  EBS Registry   & EBS HDD 500 GB & 1 & \$ 15.00 \\ \hline
  EBS PostgreSQL & EBS SSD 10 GB  & 1 & \$ 1.19 \\ \hline \hline
  \multicolumn{3}{|l|}{Monatliche Gesamtkosten} & \$ 207.84 \\ \hline
\end{tabular}
\captionof{table}
    [Berechnung der monatlichen Gesamtkosten des Clusters]
    {Berechnung der monatlichen Gesamtkosten des Clusters}
\end{flushleft}

Lässt man nun in dieser Berechnung die Faktoren weg,
die \emph{Nice-to-Have}-Features bereitstellen,
die ja in der Ausgangssituation auch nicht vorhanden waren
und wählt andere Elemente sparsam aus,
könnte diese Rechnung auch folgendermaßen aussehen:

\begin{flushleft}
\centering
\begin{tabular}{ | l | l | l | l | }
  \hline
  Zweck          & Produkt & Anzahl & Monatliche Kosten \\ \hline \hline
  Master         & EC2 t2.medium  & 1 & \$ 39.53 \\ \hline
  Worker         & EC2 t2.small   & 1 & \$ 19.77 \\ \hline
  etcd           & EC2 t2.nano    & 3 & \$ 14.94 \\ \hline
  \multicolumn{3}{|l|}{Monatliche Gesamtkosten} & \$ 74.24 \\ \hline
\end{tabular}
\captionof{table}
    [Berechnung der monatlichen Gesamtkosten des Clusters ohne Nice-to-Have-Features]
    {Berechnung der monatlichen Gesamtkosten des Clusters ohne Nice-to-Have-Features}
\end{flushleft}

Mit dem, in dieser Arbeit implementierten, Cluster fallen also rund
\$ 133.60 von \$ 207.84 an, weil \emph{Nice-to-Have}-Features umgesetzt wurden.

Um die Vergleichbarkeit zu wahren, soll also an dieser Stelle mit diesem
geringeren Betrag von \$ 74.24 weiter gerechnet werden.
Dieser Betrag steht für die fixen Kosten des Kubernetes-Clusters, wobei
die variablen monatlichen Kosten sich in etwa folgermaßen berechnen werden:

F\"ur circa 12 Stunden pro Monat mit erhöhtem Traffic,
werden circa 4 weitere Worker Instanzen benötigt werden, was Kosten
in Höhe von nur \$ 1.32 verursacht.

Um also einmal pro Monat 16.500 RPM über eine Zeitspanne von 12 Stunden
abdecken zu können und sonst mit geringer Last erreichbar zu bleiben, fallen
mit dem alten Ansatz \$ 80.- pro Monat an.
Mit dem Kubernetes Cluster sind das \$ 75.56 mit sparsamster Konfiguration.
Der Kubernetes Ansatz hat hier also \$ 4.44 weniger Kosten verursacht.

Die wahre Stärke liegt aber in der automatischen Skalierung, die, grob
geschätzt, auf 3300 RPM nur einer weitere t2.small Instanz zusätzlich
benötigt, welche stundenweise abgerechnet wird und
terminiert, sobald sie nicht
mehr benötigt wird. Deshalb wird das Kubernetes Cluster mit steigendem und
fluktuierendem Traffic auch immer effizienter und kostengünstiger als seine
Alternative.

Um zu berechnen, ab wann nun auch die Kosten der \emph{Nice-to-Have}-Features
den \emph{Break-even Point} erreichen, müsste bekannt sein, für
wie viele RPM nun welcher Server im alten Ansatz nötig wäre.
Im Rahmen dieser
Arbeit konnte dies jedoch nicht festgestellt
werden und Schätzungen diesbezüglich wären zu
ungenau. Fest steht aber, dass dieser \emph{Break-even Point} auch hier
irgendwann erreicht wäre.

Die hier beschriebenen Berechnungen sind relativ grob, was sich durch
die fehlende Erfahrung im Live-Betrieb dieses Cluster und die viele
Einflussfaktoren begr\"undet.
Deshalb soll an dieser Stelle nur
gesagt sein, dass dieses Projekt bei sparsamen Setup relativ nah und
wahrscheinlich über dem \emph{Break-even Point} liegt.
In einem Szenario, in dem mehr Traffic anfällt, hätte dieser Punkt deutlich
klarer für die finanziellen Vorteile dieses Clusters gesprochen.

\section{Sicherheit}

\subsection{TCP/IP}

Die nach außen hin verfügbaren Schnittstellen konnten alle mit TLS
und gegebenenfalls Basic-Auth abgesichert werden.
Alle weiteren Ports konnten für Firewall-Regeln, sowohl für das Internet,
als auch für das private Subnetz abgesichert werden.

Durch die Wahl einer weiteren Top-Level Domain l\"asst sich zudem
verschleiern,
unter welchen IP-Adressen sich die Instanzen befinden,
die für den User nicht
relevant sind (in diesem Fall: Admin, Master und Worker Nodes).
Dies ist jedoch hier aus Kostengründen nicht
umgesetzt worden.

Auch die SSH Daemons auf den CoreOS Instanzen und die
Kubernetes API sind mit Public-Key
Verschlüsselung ausreichend abgesichert.
Sollte es sich in der Zukunft herausstellen, dass auf
diesen Endpoints tatsächlich öfter unbefugte Anfragen eingehen,
könnten IP-Table Regeln hinzugefügt werden, welche diese dann blocken.
Im Rahmen dieser Arbeit hat sich dies jedoch nicht als nötig erwiesen.

\subsection{Pods}

Für die Ingress-Controller kann Rate-Limiting per Nginx aktiviert
werden, was die Gefahr von DDoS-Attacken verringert.

Sollte ein Angreifer einen laufenden Pod übernommen haben,
verhindern die
Docker Security Features, dass ein Ausbrechen aus diesem Pod
möglich ist \cite{dockersec}.

\subsection{Betriebssystem}

Durch die automatischen Updates von CoreOS werden Sicherheitslücken des
Betriebssystems zeitnah automatisch geschlossen.

\subsection{Kubernetes Keys und Zertifikate}

Um mit dem Kubernetes Master interagieren zu können, sind auf allen Nodes
Key- und Zertifikat-Dateien abgelegt. Sollten
diese durch einen Angriff offen gelegt
werden, hat der Angreifer volle Kontrolle über das Cluster.
Hierfür sollte bei einer Weiterentwicklung des Clusters das
\emph{Authorization Plugin} und \emph{Service Accounts} verwendet werden, sodass
dann nur ein Subset der Zugriffsrechte offen liegt.

\section{Automatische Skalierung}

Das Testen der Skalierung der Pods innerhalb des Clusters und der Nodes,
auf denen diese Pods laufen, hat sich als schwieriges Unterfangen herausgestellt.

Die Last muss für diese Tests schrittweise angehoben werden
(\quotes{Lasttest - up} \cite{cloudtest}),
damit sich das Cluster anpassen kann.

In den folgenden Punkten werden die verwendeten Tests und deren Ergebnisse
erörtert.

\subsection{Cluster-interner Test}

Die Last kann innerhalb des Clusters nachgestellt werden, indem ein oder mehrere
Pods deployed werden, die diese Last direkt auf den Kubernetes Service schicken.
Der Ingress-Controller wird hierbei umgangen.
Dies ist in diesem Fall mit einem Image passiert, welches über NodeJS
asynchrone Requests startet. Das Cluster hat sich gemäß der Vorgabe verhalten
und konnte die Test-Requests alle beantworten.

\subsection{Externes Load-Testing}

Externes Load-Testing ist ein komplexes Unterfangen, da für einen
reellen Test auch reelle Bedingungen geschaffen werden müssen.
So müsste simuliert werden, dass in manchen Fällen der TLS-Handshake
abgearbeitet werden muss in anderen ist dies bereits geschehen und
Client und
Server können bereits über schnellere synchrone
Verschlüsselung Daten austauschen.
In anderen Fällen sind Requests unterschiedlich
komplex, sodass
eine höhere Last im Webserver oder in der Datenbank entsteht.

Im Rahmen dieser Arbeit sind nur einfache immer gleiche Requests, welche
API-Anfragen, mit Datenbank-Lesezugriff anstoßen, getestet worden.

Zudem ist das Erzeugen der Last selbst ein Problem, denn sollte diese Last
zum Beispiel vom Entwickler-Rechner generiert werden, ist die maximale Last
von Faktoren abhängig, wie die technischen Spezifikationen
dieses Rechners, der Performance des verwendeten Tools oder
der Bandbreite der Internetverbindung.

In den Tests wurde wieder der bereits erwähnte NodeJS-Anwendung benutzt,
sowie \emph{Apache Benchmark}. Die Last-Spitzen von 16.500 RPM haben in beiden
Fällen zu
Response-Times von durchschnittlich 7,23 Sekunden geführt.
Diese sind jedoch zum größten Teil durch die Internetleitung und die
Ausstattung des verwendeten Laptops entstanden und taugen nicht für ein
abschließendes Fazit. Was man aber erkennen kann an diesem Resultat, ist,
dass
alle gestarteten Requests auch beantwortet werden konnten und die
Last an diesem
Punkt nicht dazu geführt hat, dass Requests verworfen werden mussten.

\subsection{Load-Testing im Live Betrieb}

Um ein abschließendes Bild der Performance des Clusters zu bekommen,
wurde dieses mit dem Legacy-System ausgetauscht und hat während der
Lastspitzen die eingehenden Requests von echten Usern verarbeitet.
Da an diesem Tag maximal nur circa 11.000 RPM angefallen sind, ist die
durchschnittliche Lastspitze der Anforderung nicht erreicht gewesen.

Folgende Beobachtung konnten trotzdem gemacht werden:
Die Skalierung der API-Pods und der darunter liegenden Nodes funktioniert
genau so, wie es im Vorfeld geplant war und hat keine schweren
Performance-Probleme offenbart.
Trotzdem war die insgesamte Performance eingeschränkt
durch folgende Flaschenhälse:
\begin{description}
  \item[Datenbank-Pod:] Dieser ist in diesem Szenario nur ein einzelner Pod,
  der sich bei hoher Last nicht weiter horizontal skalieren kann.
  Zur Abhilfe sollte auch dieser in Zukunft in einem
  High-Availability-Modus mit Replicas und gegebenenfalls Shards betrieben
  werden \cite{Vohra}. Auch wäre denkbar die Queries, die die API Pods absetzen, weiter
  zu optimieren.
  \item[Ingress-Controller:]
  Vor allem durch die TLS-Termination entstehen hier hohe CPU-Lasten.
  Eine weiteres Nadelöhr könnte hier im Laufe der Zeit auch die
  Netzwerk-Performance werden.
  Mögliche Lösungen wären:
  \begin{itemize}
    \item Vertikale Skalierung der Hardware der Ingress-Instanz
    \item Eine weitere
    Ingress Instanz, welche über DNS-Loadbalancing Last annimmt
  \end{itemize}
\end{description}

\section{Vendor Lock-in}

Um einschätzen zu können, wie portabel das Cluster tatsächlich ist, soll
darauf eingegangen werden, welche Komponenten bei einem Umzug auf einen
anderen Iaas-Anbieter verändert werden müssten. Als Beispiel soll ein Umzug
in die Google Cloud dienen.

\subsection{Terraform Konfiguration}

Die vorhandenen
Terraform AWS Ressourcen müssen mit den äquivalenten Google Cloud Ressourcen
ausgetauscht werden.
Funktional bietet die Google Cloud alle Features, die für dieses Cluster
benötigt werden,
sodass hier keine Workarounds notwendig sind.
Was aber als zusätzlicher Arbeitsaufwand zu Buche schlägt, ist die
Änderung der neuen Konfigurationselemente zu denen der Google Cloud.
Diese haben andere Namen und andere Felder über die die jeweiligen Werte
definiert werden.

\subsection{Cloud-Config}
Die Cloud-Config Dateien können 1 zu 1 übernommen werden. Da CoreOS bei beiden
Anbietern in gleicher Weise arbeitet, ist davon auszugehen, dass diese
unabhängig vom IaaS-Anbieter immer gleich funktioniert.

\subsection{Kubernetes Manifeste}
Auch Kubernetes kann auf beiden Plattformen in derselben Version und mit
denselben Features deployed werden. Einzig die EBS Volumes und
die Verknüpfung vom \emph{Cluster Autoscaler Pod} zur \emph{Auto Scaling Group} müssen
gemäß dem Google Cloud \"Aquivalent abgeändert werden.

\section{Ausfallsicherheit}

Das Cluster, wie es in dieser Arbeit implementiert wurde, stellt eine Ausfallsicherheit
bereit für folgende Fälle:
\begin{description}
  \item[Pod fällt aus:] Wird erneut gestartet
  \item[Worker Node fällt aus:] Die Pods auf diesem werden auf einer
  anderen Node gestartet
  über die \emph{Auto Scaling Group} wird eine neue Worker Node gestartet.
\end{description}

Die automatische Wiederherstellung wurde getestet, indem diese Instanzen
oder Pods terminiert wurden.
\hfill
\\
Fälle, die mit dieser Implementation nicht ausfallsicher sind:

\begin{description}
  \item[Availability Zone:]
  Sollte bei AWS die Availability Zone ausfallen, auf dem das Cluster liegt,
  ist damit das komplette Cluster nicht mehr zu erreichen. Eine Lösung, welche
  aber eben auch die Kosten in die Höhe treibt, ist, das Cluster über zwei
  Availability Zones auszubreiten. So kann die übrig bleibende Availability Zone
  die Aufgaben der ausgefallenen Availability Zone übernehmen.
  \item[Master, Ingress und Testing Node:]
  Hier wurde aus Kostengründen keine Redundanz geschaffen.
  Als Lösung könnten hier jeweils Redundanzen, am besten in
  einer weiteren Availability Zone, geschaffen werden.
\end{description}
\hfill
\\
Das Cluster ist also nur in einem gewissen Rahmen ausfallsicher. Mit
größerem Budget könnte der Grad der Ausfallsicherheit jedoch erhöht werden.

\section{Disaster Recovery}

Da im Vorfeld nicht bekannt ist, um welchen Ausfall es sich genau handelt,
lässt sich hier nur folgende generelle Aussage treffen:

Dadurch, dass die Infrastruktur in Terraform definiert wurde, kann diese
auch mit Terraform ganz oder teilweise
wiederhergestellt werden.

Das \emph{Recovery Time Objective} (RTO) \cite{wood2010disaster} ist also
im schlechtesten Fall
die Zeit, die Terraform benötigt um alle Infrastruktur-Elemente
mit \code{terraform destroy} oder \code{terraform taint} zu löschen
und mit \code{terraform apply} wieder aufzusetzen.
Mit der finalen Komplexit\"at dieses Clusters, liegt diese Zeitspanne bei
rund 21 Minuten.

Das \emph{Recovery Point Objective} (RPO) \cite{wood2010disaster} hängt allein von der Häufigkeit der
Backups ab und kann je nach Bedarf individuell bestimmt werden. In dieser Arbeit
ist dieser Wert nur für die PostgreSQL-Datenbank relevant.
Das ausgelagerte EBS Volume stellt eine RPO von 0 Sekunden sicher. Sollte jedoch
auch das EBS ausfallen, ist die RPO abhängig von der Häufigkeit, mit der
Backups außerhalb des Cluster gemacht werden, was aber im Rahmen
dieser Arbeit nicht implementiert wurde.

\section{Deployment Pipeline}

Durch die Schnittstellen der Deployment Komponente
und den Aufbau des Clusters war es möglich, mittels
eines Skriptes automatische Deployments umzusetzen.
Die Zeit zwischen Commit und dem Release in der jeweiligen Umgebung ist
damit besonders kurz und Images mit Domain-Logik müssen nicht außerhalb
des Clusters existieren, was die Sicherheit dieser Daten verbessert.

\section{Open Source statt proprietärer Software}
Ausnahmslos alle Komponenten, die in dieser Arbeit umgesetzt wurden, sind
Open Source Software.

\section{Self Hosted statt externe Provider}
Alle Komponenten dieses Clusters können im Cluster selbst gehostet werden.
Einzige Ausnahme ist hier das Docker Image, welches das Bootstrapping und die
Deployment-Pipeline abbildet. Da dies aber kostenfrei ist, kann diese
Ausnahme hingenommen werden.

\section{Logging, Monitoring und Alerting}

Für alle diese Anforderugen konnten Komponenten
in diesem Cluster umgesetzt werden. Die einzelnen
Komponenten sollten jedoch zusätzlich noch weiter für diesen Use Case
konfiguriert werden, um die zur Verfügung stehenden Features entsprechend
nutzen zu können.

\section{The 12-Factor App}

Die Regeln aus dem \quotes{The 12-Factor App}-Manifest, lassen sich alle
anwenden und werden teilweise auch von Kubernetes so erzwungen.
