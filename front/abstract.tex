\chapter{Abstract}

Das Bereithalten von Applikationen in der Cloud
ist mit einer Reihe von komplexen Zusammenh\"angen und Abl\"aufen
verbunden. Mit Tools wie Ansible, Puppet oder Chef ist die Administration
oft sehr kleinteilig, statisch und oft nicht deklarativ genug.

Das Container Orchestration Tool \quotes{Kubernetes} bietet eine Abstraktion
der Host-Systeme an. Es fasst mehrere Server in einem Cluster
zusammen und bietet \textbf{eine} Schnittstelle f\"ur das Management
\textbf{aller} Server an.

Mit der Abstraktion des Host-Systems und der Abstraktion der
Applikations-Umgebung durch Docker- oder rkt-Container, ergibt sich ein
neues Paradigma f\"ur den Betrieb eines Clusters in der Cloud.

Kubernetes vereint L\"osungen zur Verf\"ugung f\"ur Standardvorg\"ange,
wie Deployments oder Skalierung. Durch zus\"atzliche
Module kann der Funktionsumfang weiter ausgebaut werden.

Die Installation von Kubernetes ist aber aufgrund der großen Zahl an
unterschiedlichen Herangehensweisen mit einigen Schwierigkeiten verbunden.
Speziell die Kombination aus Terraform in Verbindung mit CoreOS auf
Amazon Web Services Infrastruktur ist dabei zwar eine naheliegende Kombination,
jedoch aktuell nicht ausreichend erprobt und dokumentiert.

Diese Arbeit nimmt sich dieser Herangehensweise an und er\"ortert, wie
eine Installation dieses Clusters, sowie der weiteren n\"otigen Komponenten
umgesetzt werden kann. Um einen besseren Einblick in den Einsatz des Clusters
unter realen Bedingungen zu gewinnen, wird ein existierendes
Backend auf dieses Cluster migriert.

Auf Grundlage von bestimmten Anforderungen wird ausgewertet, inwiefern
das implementierte System diesen Anforderungen gerecht wird.
