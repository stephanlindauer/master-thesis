\chapter{Glossar}
\label{ch:Glossar}

\begin{description}
  \item[Development Operations (DevOps): ]
  In diesem Bereich der IT geht es
  darum, die Entwicklung von Software besser mit den operativen Aspekten
  zu vereinen.

  \item[Infrastructure as a Service (IaaS): ] Infrastruktur wird bei einem externen
  Dienstleister angemietet. So werden Kosten f\"ur den Betrieb eines eigenen Datencenters
  eingespart. Beispiele: Digital Ocean, Hetzner und Google Cloud.

  \item[Platform as a Service (PaaS): ] Eine komplette Cloud-Plattform wird von einem
  externen Dienstleister angemietet. Die Administration der Plattform und der darunter
  liegenden Hardware und Software wird so
  outgesourced. Beispiele: Elastic Beanstalk und Heroku.

  \item[Software as a Service (SaaS): ] Es wird eine spezielle Anwendung outgesourced.
  Alle Aspekte des Betriebes dieser Software werden von Dienstleister \"ubernommen.
  Beispiele: Datadog, Pingdom und Zendesk.

  \item[Quorum: ] Die Mindestanzahl an Members in einem Cluster, um eine Entscheidung
  treffen zu k\"onnen, wie
  zum Beispiel bei einer \quotes{Leader election}.

  \item[Workload: ] Meint Prozesse, die entweder durchgehend laufen, wie Webserver,
  aber auch Prozesse, die einen bestimmte Arbeit verrichten und dann terminieren.

  \item[Orchestrierung: ] Ist die Platzierung und das Management von Workloads
  in einem Server-Cluster.

  \item[Node/Instanz: ] Beide Begriffe stehen f\"ur \quotes{Server}.
  \quotes{Instanz} ist dabei der Begriff der von AWS gepr\"agt wurde. \quotes{Node}
  ist der Begriff den Kubernetes verwendet.

  \item[Deployment: ] Ist der Vorgang, der daf\"ur sorgt, dass Software
  ihrem Einsatzzweck zugef\"uhrt wird.

  \item[Single Point of Failure: ] Einzelnes Element innerhalb eines Systems,
  das bei einem Ausfall das komplette System in Mitleidenschaft ziehen w\"urde.

\end{description}
